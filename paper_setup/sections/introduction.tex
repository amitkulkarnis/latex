
\section{Inleiding}
In dit practicum zullen we {\em opnieuw} kennis maken met de FPGA 
({\em Field Programmable Gate Array}). Een FPGA werd reeds gebruikt in de 
oefeningen van het opleidingsonderdeel ``Digitale Elektronica'', waar de programmering 
gebeurde op logisch niveau (cfr.\ het model van Gajski). In deze oefening 
is het de bedoeling VHDL-code te schrijven op {\em Register-Transferniveau}
(RT-niveau). 

Het doel van deze oefening, die over twee sessies loopt, is:
\begin{itemize}
\item het programmeren in VHDL op te frissen en ontwerptechnieken op RT-niveau te leren 
toepassen, 
\item door een een\-vou\-dig algoritme op ver\-schil\-len\-de ma\-nie\-ren te 
im\-ple\-men\-te\-ren een inzicht te verwerven in de flexibiliteit van een FPGA, en
\item een link te leggen tussen digitaal ontwerp in VHDL en mechatronica.
\end{itemize}

\section{Opgave}

Als illustratie zullen we een PI-regelaar in discrete tijd
implementeren.  Het PI-regelalgoritme werkt als volgt:

Initialisatie:

$$O_1 = 0$$
$$F_1 = K_s \cdot K_i$$
$$C = K_s \cdot K_p$$

In regime:

$$Error = SP - PV$$
$$O_1 = OldO_1 + Error$$
$$MV = (F1 \cdot O1 + C \cdot Error)$$

De discrete regelparameters ($C$ en $F_1$) zijn afgeleid van de
klassieke parameters $K_s$, $K_p$ en $K_i$.  Register $O_1$ fungeert
als integrator, $OldO_1$ is de vorige waarde van $O_1$.  Verder heeft
de regelaar ingangen $SP$ (set point, wenswaarde) en $PV$ (process
value, proceswaarde).  De uitgang van de regelaar is $MV$ (manipulated
variable).

Voor deze practica maken we gebruik van Aldec Active-HDL voor
simulatie.  Wie dat wenst kan gebruik maken van Sigasi HDT voor het
editeren van VHDL. Een licentiesleutel voor gebruik in de PC-klassen
is beschikbaar op minerva, wie HDT op een priv\'e PC of laptop wil
installeren moet zelf een studentenlicentie aanvragen op {\tt
www.sigasi.com} .  Uitleg over het gebruik van deze pakketten vind je
in het appendix.


\subsection{Combinatorische implementatie}
{\em Voorziene tijd: 10 minuten} \\
{\em Map: opgave1}

De eerste implementatie van het PI-algortime is puur combinatorisch
(figuur \ref{comb}).  De VHDL-code voor dit onderdeel krijg je
cadeau. Bestudeer de code en schenk de nodige aandacht aan o.a.\
typeconversies en het aantal bits waarmee de diverse waarden
voorgesteld worden.  Wat zou er aan de code veranderen mocht je een
proces willen gebruiken i.p.v.\ concurrente assignaties?  Welke
signalen zouden er in de sensitiviteitslijst voorkomen? Niet
implementeren, alleen de belangrijkste zaken vermelden.

%% \begin{figure}[htb]
%% \begin{center}
%% \epsfxsize=8cm
%% \epsfbox{picomb.eps}
%% \caption{PI-regelaar: combinatorisch}
%% \label{comb}
%% \end{center}
%% \end{figure}

\subsection{Combinatorisch, met registers}
{\em Voorziene tijd: 20 minuten} \\
{\em Map: opgave2}

In de tweede implementatie worden registers (flip-flops) toegevoegd
v\'o\'or en na de berekening, zoals in figuur \ref{buff}.

%% \begin{figure}[htb]
%% \begin{center}
%% \epsfxsize=10cm
%% \epsfbox{pibuf.eps}
%% \caption{PI-regelaar: combinatorisch, met registers}
%% \label{buff}
%% \end{center}
%% \end{figure}

Kopieer de VHDL-code uit het vorige onderdeel. Pas de code aan zodat
ze overeenkomt met het schema uit figuur \ref{buff}. Voorzie alle
registers van een synchrone reset.  Waarom worden er twee flipflops
geplaatst tussen start en ready?

Simuleer je code met ActiveHDL. Hoe groot is de latentie (aantal
klokcycli dat de {\em berekening} duurt, m.a.w.\ vanaf de uitgang van
de ingangsregisters tot de uitgang van het resultaat-register)?

\subsection{Gepijplijnd}
{\em Voorziene tijd: 20 minuten} \\
{\em Map: opgave3}

Een volgende implementatie bestaat uit het pijplijnen van het vorige
ontwerp.  Op een aantal plaatsen worden registers ingevoerd (zie
figuur \ref{pipe}).  Het {\em kritisch pad} zal nu korter zijn zodat
de kloksnelheid hoger kan, anderzijds duurt de berekening nu meerdere
klokcycli. Merk tevens op dat een aantal ingangen niet van een
register voorzien zijn, we doen dit omdat we veronderstellen dat die
ingangen niet veranderen tijdens de berekening.

%% \begin{figure}[hbt]
%% \begin{center}
%% \epsfxsize=12cm
%% \epsfbox{pipipe.eps}
%% \caption{Gepijplijnde PI-regelaar}
%% \label{pipe}
%% \end{center}
%% \end{figure}

Kopieer je VHDL-code uit de vorige opgave, en maak de nodige
uitbreidingen.  Let op i.v.m.\ de correctheid van het {\em
ready}-signaal.  Simuleer de schakeling met ActiveHDL en noteer de
latentie.

\subsection{Inwendig register voor $O_1$}
\label{sec:opgave_oint}
{\em Voorziene tijd: 20 minuten} \\
{\em Map: opgave4}

%% \begin{figure}[hbt]
%% \begin{center}
%% \epsfxsize=12cm
%% \epsfbox{pioint.eps}
%% \caption{PI-regelaar: $O_1$-register in de schakeling opgenomen}
%% \label{o1int}
%% \end{center}
%% \end{figure}

In het PI-algoritme is $OldO_1$ telkens de waarde die in de vorige
tijdstap berekend werd voor $O_1$. Het lijkt dan ook logisch $O_1$ te
implementeren als een intern register zoals in figuur \ref{o1int}. Een
multiplexer (mux) zorgt ervoor dat een nieuwe waarde in het register
geschreven wordt wanneer de controle-ingang van de mux '1' is; als het
controle- signaal '0' is wordt de oude waarde opnieuw geschreven
m.a.w. het register behoudt zijn waarde.

Kopieer opnieuw uw VHDL-code van het vorige onderdeel en maak de
nodige aanpassingen.  De mux wordt gecontroleerd door een signaal dat
afgeleid is van het start-signaal: zorg voor de correcte vertraging
zodat de waarde van $O_1$ in het register opgeslagen wordt.


\subsection{Datapad en gemicroprogrammeerde controle-automaat}
\label{datapadsexy}
{\em Voorziene tijd: 70 minuten} \\
{\em Map: opgave5}

In de vorige onderdelen werd voor elke bewerking aparte hardware
ge\-\"{\i}n\-stan\-tieerd.  Voor niet al te ingewikkelde berekeningen
waar snelheid primordiaal is, kan dit een goede aanpak zijn. Voor
berekeningen die niet zeer tijdskritisch zijn komt deze aanpak neer op
een verkwisting van beschikbare hardware, en voor complexe
berekeningen treedt het fenomeen van {\em combinatorische explosie}
op: de schakeling wordt snel enorm groot.

%% \begin{figure}[hbtp]
%% \begin{center}
%% \epsfxsize=12cm
%% \epsfbox{datapath.eps}
%% \caption{PI-regelaar: datapad}
%% \label{datapad}
%% \end{center}
%% \end{figure}

Om dit op te lossen kan men gebruik maken van een datapad met een
beperkte hoeveelheid rekeneenheden, in combinatie met een
controle-automaat om dat datapad te besturen. Het datapad bestaat uit
een {\em relatief beperkt} aantal rekeneenheden, aangevuld met
registers om (tussen)resultaten bij te houden. Multiplexers in het
datapad zorgen ervoor dat gegevens tussen de rekeneenheden en de
registers doorgegeven worden (zie ook cursussen computerarchitectuur).

Voor deze opgave beperken we het datapad tot \'e\'en aftrekker,
\'e\'en vermenigvuldiger en \'e\'en opteller. Het datapad ziet er dan
uit zoals in figuur \ref{datapad}; de signalen {\tt K1} tot {\tt K4}
zijn controlepunten die aangestuurd worden de controle-automaat.
Tabel \ref{microsingle} beschrijft het verloop van de berekening.

\begin{table}
\caption{Werking van toestandsmachine en datapad
\label{microsingle}
}
{\tiny
\begin{tabular}{c c c c c }
Klokcyclus & Aftrekker & Opteller & Vermenigvuldiger & Resultaat \\
\hline
0 \\
1  & $Error=SP-PV$ \\
2  & & $O1=OldO1+Error$ & $A1=C*Error$\\
3  & &                  & $A1=F1*O1$      \\
4  & & $MV=A1+A2      $ & &  \\
5  & &                  & & Klaar! \\

\end{tabular}
}
\end{table}


Vul de VHDL-code van het datapad aan. De controle-automaat en de
microcode-tabel zijn reeds ingevuld. Controleer de correctheid van het
datapad door het geheel te simuleren met ActiveHDL.  Hoeveel bedraagt
de latentie nu?

\subsection{Datapad en Moore-controle-automaat}
{\em Voorziene tijd: 70 minuten} \\
{\em Map: opgave6}

In de vijfde implementatie van de PI-regelaar behouden we het datapad
van de vorige implementatie, en wordt de controle-automaat vervangen
door een {\em Moore}-machine die we in VHDL beschrijven. Deze
controle-automaat moet uitwendig hetzelfde gedrag vertonen als de
gemicroprogrammeerde versie.

Schrijf eerst uit welke toestanden de automaat heeft, en welke de
uitgangen (controlepunten) zijn die bij elke toestand behoren. Je kan
dit afleiden uit de inhoud van de microcode-ROM, of uit de simulatie
van het vorige onderdeel. Duid aan of sommige uitgangen {\em don't
care}'s bevatten.

Vul vervolgens de VHDL-code van de controle-automaat aan. Het datapad
kopieer je uit de vorige opgave. Controleer de correctheid m.b.v.\
ActiveHDL.


\subsection{Getalbereik en fixed point getallen}
{\em Voorziene tijd: 70 minuten} \\
{\em Map: opgave7}

In deze oefening zullen we nagaan wat de optimale voostelling is van
de getallen in het PI-algoritme.  Tot nu toe zijn we er blindelings
van uitgegaan dat alle getalwaarden voorgesteld konden worden als
16-bit gehele getallen. Daardoor is het niet gegarandeerd dat de
schakeling correct zal werken. Indien er te weining bits zijn om een
getal voor te stellen wordt het resultaat minder nauwkeurig als de
minstbeduidende bits wegvallen, of er kan overflow optreden (en het
resultaat zal totaal fout zijn) als de meestbeduidende bits
wegvallen. Wanneer er onnodig veel bits gebruikt worden, worden de
operatoren breder, waardoor de schakeling onnodig groot en traag
wordt.

Daarnaast kan het ook voorkomen dat de schakeling kommagetallen moet
verwerken, waarvoor het gehele getaltype niet zomaar bruikbaar is.
Vermits vlottende-kommagetallen (floating point) niet zeer geschikt
zijn voor logische synthese, zullen we voor deze opgave gebruik maken
van vaste-kommagetallen (fixed point). Achteraf zullen we de
vaste-kommagetallen op hun beurt op gehele getallen afbeelden.  Uitleg
over vaste-kommagetallen vind je o.a.\ op \\ 
\small{\url{http://www-inst.eecs.berkeley.edu/~cs61c/sp06/handout/fixedpt.html}}.

\normalsize{Eerst zullen we analyseren hoe we elk van de getallen in het ontwerp
(ingangen, registers, tussenresultaten \ldots) kunnen voorstellen. We
gaan uit van een {\em signed} fixed point notatie, voor elk getal
wordt dus een tekenbit voorzien. Nu moeten we voor elk getal het
totaal aantal bits en het aantal bits na de komma bepalen. }

Als basis gebruiken we het circuit uit paragraaf
\ref{sec:opgave_oint}.  Van de ingangen $PV$ en $SP$ veronderstellen
we dat ze een afstand voorstellen die kan vari\"eren van +1m tot -1m,
en dat we die afstand tot op 1mm nauwkeurig willen voorstellen. Voor
de uitgangen kies je hetzelfde totaal aantal bits; het aantal bits na
de komma kan verschillen. Voor interne tussenresultaten neem je in het
totaal 4 bits meer dan voor de in- en uitgangen, tenzij wanneer dat
niet zinvol is. voor de regelconstanten $F1$ en $C$ kies je aantallen
bits zodat je waarden in de orde van -.001 \ldots .001 kan voorstellen
met max.\ 1\% relatieve fout. {\em Bepaal voor elk van de ingangen,
inwendige signalen, registers en uitgangen de aantallen bits (totaal
en fractioneel) van de fixed point-getallen}.

In VHDL is de ondersteuning van fixed-point nog in ontwikkeling. Er
bestaat een package met signed en unsigned fixed point datatypes \\ ({\tt
ieee\_proposed.fixed\_pkg}), maar dat wordt niet door alle tools
ondersteund. In de plaats daarvan kunnen ook de signed en unsigned
datatypes gebruikt worden. De bewerkingen zijn ongeveer dezelfde,
alleen moet de ontwerper dan zelf bijhouden waar de komma staat en
waar nodig operandi aligneren. Het verschuiven van de komma komt neer
op het vermenigvuldigen van de getallen met een macht van 2. {\em Pas
de VHDL-code uit opgave 4 aan voor de eerder gevonden
getalrepresentaties: zowel de regelaar als de testbank. Stel
$F1=0.0001$ en $C=-0.0004$. Controleer door simulatie dat de
schakeling correct werkt}.

\newpage

\appendix
\section{Gebruik van Aldec ActiveHDL}

Aldec ActiveHDL werd reeds in de cursus Digitale Elektronica gebruikt,
wat hier volgt zijn nog enkele aandachtspunten.

Om de bestanden te organiseren maak je in ActiveHDL \'e\'en workspace
aan (op je {\tt H:}-schijf) en vervolgens {\em voor elke opgave een
nieuw design} in die werkruimte. Bij het aanmaken van een nieuw
ontwerp kies je de optie {\tt Add existing Resource Files}. Selecteer
alle VHDL- en geheugenfiles uit de opgavemap (extensies {\tt .vhd} en
{\tt .mif}). Hou er rekening mee dat ActiveHDL die bestanden naar het
project {\em kopieert}, aanpassingen moet je dus maken in de
ActiveHDL-workspace en niet op de originele locatie (waar je het
zip-archief uitgepakt hebt).

Als je van Sigasi HDT gebruik wil maken moet je eerst een design
aanmaken in ActiveHDL en daarna dat ontwerp in HDT importeren -- niet
omgekeerd.

\section{Gebruik van Sigasi HDT}

Sigasi Hardware Development Tool (HDT) wordt ontwikkeld door een
Gentse KMO.  Deze editor analyseert de VHDL-code en kan op
verschillende gebieden hulp bieden bij de ontwikkeling van
hardware. Zo worden bijvoorbeeld syntaxfouten en niet gedeclareerde
symbolen aangeduid, kan je navigeren doorheen de code, indenteren,
{\em refactoring} toepassen...

HDT is niet ge\"{\i}nstalleerd in de PC-klas, maar je kan dit zelf
snel installeren. De zip-file met HDT vind je op minerva. Pak dit bij
voorkeur uit op de {\tt H:}-schijf (zodat je het de volgende keer niet
opnieuw moet installeren). Alle bestanden komen in de map {\tt sigasi}
terecht. Om HDT op te starten dubbelklik je {\tt sigasi.exe}.

De eerste keer moet je een paar zaken instellen. Eerst moet je een
workspace kiezen: accepteer wat ingevuld is en klik OK. Dan krijg je
een welkom-pagina te zien die je mag sluiten (get started). Daarna
moet je de licentie instellen: navigeer naar {\tt window $\rightarrow$
preferences $\rightarrow$ VHDL $\rightarrow$ License Key} en plak of
typ de licentiesleutel {\tt 27011@deugniet.elis.UGent.be} in het
tekstvenster. Klik daarna op {\tt Apply} en vervolgens {\tt OK}.

Voor elk deel van de opgave moet je een apart project aanmaken in
hdt. Omdat ActiveHDL van een eigen mappenstructuur gebruik maakt moet
je eerst een project in Aldec aanmaken (zie boven) en dan pas die
bestanden in HDT importeren. Procedure voor HDT: {\tt File
$\rightarrow$ New $\rightarrow$ Point to existing VHDL project}.
Navigeer naar de {\tt src}-map van het ActiveHDL-project en klik {\tt
Finish}. Het nieuw aangemaakte project heeft de naam {\tt src}
gekregen, dit {\em moet} je veranderen door in de Project Explorer
(linker paneel) het project te selecteren en dan via {\tt F2} of {\tt
rechtsklik $\rightarrow$ Rename} een {\em zinvolle} naam te geven.

Voor delen 5 en 7 van de opgave zal je zien dat er 2 fouten in de code
van het rom-geheugen gerapporteerd worden.  HDT heeft immers geen
beschrijving van de geheugenmodules van Xilinx, dus moeten we een
verwijzing maken naar de plaats waat HDT die kan vinden.  Open het
bestand met de fout erin en klik op het {\em quick fix}-icoontje links
van de regel {\tt library XilinxCoreLib}.  Kies de optie {\tt Define
undefined library}. In het popup-venster selecteer je de
niet-gedefinieerde locatie en klik {\tt edit}. Kies in de volgende
pop-up {\tt Select external folder} en navigeer naar {\tt
C:$\backslash$Xilinx91i$\backslash$vhdl$\backslash$src$\backslash$XilinxCoreLib}.
Daarmee zouden beide fouten m.b.t. het ram-geheugen opgelost moeten
zijn.

%% \section{Gebruik van Xilinx ISE}

%% Xilinx ISE is een synthesetool, d.i.\ een software-pakket dat een gedragsbeschrijving in VHDL
%% vertaalt in een netlijst met basiscomponenten uit de gekozen technologie (in het geval van
%% ISE: bouwblokken van een Xilinx-FPGA) en de verbindingen tussen die componenten.
%% ISE werd reeds gebruikt in de practica bij het vak ``digitale elektronica''.

%% Om een netlijst te synthetiseren in ISE ga je als volgt te werk:

%% \begin{itemize}

%% \item open project navigator 

%% \item maak een nieuw project aan voor elk onderdeel van de opgave:

%% \begin{itemize}
%% \item {\tt file} $\rightarrow$ {\tt new project\ldots} 
%% \item Vul in: eerst de Project Location (bvb.\ {\tt C:$\backslash$temp}), dan Project Name ({\tt opgave{\em x}}), 
%% Source Type is {\tt HDL}.
%% \item In het volgende scherm: Family {\tt Virtex2P}, Device {\tt XC2VP30}, \\ Speed Grade {\tt -7}.
%% \item Create a new source: klik {\tt next}.
%% \item Add existing sources: klik op Add Source, en voeg alle vhdl- , ucf- en xco-files toe behalve de testbank 
%% (VHDL uit de ActiveHDL-map, 
%% UCF en XCO uit het uitgepakte zip-archief). Voor de VHDL-code moet je de optie {\tt Copy to Project} afzetten.

%% \end{itemize}

%% \item klik in het venster linksboven (module view) op de de 
%% topniveau-module {\tt piregelaar-\ldots};
%% \item in het venster daaronder (process view), klik rechts op {\tt 
%% implement design} en kies {\tt run}.
%% \end{itemize}

%% De VHDL-code wordt dan gesynthetiseerd, en de netlijst wordt geplaatst 
%% (d.w.z.\ de componenten uit de netlijst krijgen een plaats toegewezen op
%% de chip) en gerouteerd (de verbindingen worden gelegd). Waar er gebruik 
%% gemaakt is van rom-geheugens (opgave 5 en 7) moet dat geheugen de eerste 
%% keer opnieuw gegenereerd worden (de software zal vragen om dit te doen). 

%% In het console-venster kan je zien welke stappen de software genomen 
%% heeft, en hoeveel flip-flops enz.\ er gebruikt worden (tabel aanvullen).
%% In het Fpga Design Summary-venster kan je allerlei rapporten selecteren,
%% o.a.\ het {\em mapping}-rapport dat een overzicht van de gebruikte 
%% FPGA-cellen bevat en het {\em static timing}-rapport.

%% Wanneer geen eisen qua timing opgelegd worden, doet de software geen 
%% enkele moeite om de timing te optimaliseren. Door een klokperiode op te 
%% geven (of andere timing-{\em constraints}) kan de software gedwongen 
%% worden om plaatsing en routering aan te passen zodat aan die eisen voldaan 
%% wordt. Wanneer zeer strikte eisen opgelegd worden -- die niet 
%% realiseerbaar zijn -- zal de software dit rapporteren en de 
%% optimalisatiepogingen stop zetten. Het snelste circuit wordt gevonden 
%% wanneer eisen opgelegd worden die net niet haalbaar zijn: de software doet 
%% dan een maximale inspanning.

%% Na de eerste synthese kan je de reeds gehaalde klokperiode uit het {\em 
%% timing report} aflezen en een kleinere waarde opleggen om de optimalisatie
%% te laten werken. 

%% \begin{itemize}
%% \item selecteer de topniveau-entity (module view);
%% \item {\tt process view $\rightarrow$ User constraints $\rightarrow$
%% Create timing \\ constraints};
%% \item vul de vooropgestelde klokperiode in, zonder eenheden (eenheid is 
%% ns), daarna {\tt TAB};
%% \item {\tt File $\rightarrow$ Save}, {\tt File $\rightarrow$ Exit} in 
%% constraints editor;
%% \item laat de synthesestap opnieuw lopen ({\tt Implement design 
%% $\rightarrow$ Run}).
%% \end{itemize}

%% In het console-venster kan je zien welke stappen de router doorloopt.
%% Wanneer je ziet dat fase 8 en volgende gestart worden, weet je dat de 
%% uitgebreide optimalisatie gestart werd. Eventueel kan je de 
%% optimalisatie een paar keer herhalen om ``het onderste uit de kan''
%% te halen.

